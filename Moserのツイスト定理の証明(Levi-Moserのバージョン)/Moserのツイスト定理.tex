\documentclass[a4paper]{ujarticle}
\usepackage[margin=15mm]{geometry}
\usepackage{graphicx}
\usepackage[dvipdfmx]{color}
\usepackage{amsmath,amssymb,amsthm}
\usepackage{mathtools}
\mathtoolsset{showonlyrefs=true}
\usepackage{bm} %太字
\usepackage{tcolorbox} %定理環境のやつ
\usepackage{physics}
\tcbuselibrary{breakable, skins, theorems}
\usepackage{enumerate} %箇条書き
\usepackage{ulem} %下線
\usepackage{mathrsfs} %花文字
\usepackage{url}

%コマンド
\newcommand{\red}[1]{\textcolor{red}{#1}} %赤文字
\newcommand{\vecn}[1]{({#1}_1, \cdots, {#1}_{n})}
\newcommand{\inner}[2]{(#1, #2)}
\newcommand{\R}{\mathbb{R}}
\newcommand{\C}{\mathbb{C}}
\newcommand{\Q}{\mathbb{Q}}
\newcommand{\Lpn}[1]{\|#1\|_{L^{p}}}
\newcommand{\esssup}[1]{\mathop{\mathrm{ess.sup}}_{#1}}
\newcommand{\cl}[1]{\bar{#1}}
\newcommand{\pd}[2]{\partial_{#1}^{#2}}
\newcommand{\alev}{\ \mathrm{a.e.}}
\newcommand{\sgn}{\mathrm{sgn}}
\newcommand{\weak}{\ \mathrm{(weakly)}}
\newcommand{\Div}{\mathrm{div}}
\newcommand{\Grad}{\mathrm{grad} }
\newcommand{\Rot}{\mathrm{rot} }


\numberwithin{equation}{section}
\mathtoolsset{showonlyrefs=true}
\newcommand{\rme}{\mathrm{e}}

\theoremstyle{definition}
\newtheorem{definition}{Definition}
\newtheorem{theorem}{Theorem}
\newtheorem{proposition}{Proposition}
\newtheorem{example}{Example}
\newtheorem{lemma}{Lemma}
\newtheorem{cor}{Corlollary}
\newtheorem{remark}{Remark}
\newtheorem{conj}{Conjecture}
\newtheorem{axiom}{Axiom}
%%証明環境
\makeatletter
\renewenvironment{proof}[1][Proof]{\par
  \pushQED{\qed}%
  \normalfont \topsep6\p@\@plus6\p@\relax
  \trivlist
  \item\relax
  {\bfseries
  #1\@addpunct{.}}\hspace\labelsep\ignorespaces
}{%
  \popQED\endtrivlist\@endpefalse
}

\title{Moserのツイストの定理}
\begin{document}
\date{2025年9月11日}
\author{まるげり}
\maketitle
    Levi-Moser\cite{LM01}のノート. 
    面積保存ツイスト写像の母関数に解析性を課すが, ツイスト定理の証明としては一番読みやすいと思う. 

    \section{背景: 面積保存ツイスト写像, 母関数}
        アニュラス$\mathbb{A} = \mathbb{S}^1 \times \mathbb{R}$上の面積保存ツイスト写像$\varphi(x_1, y_1) = (x_2, y_2)$とは, 
        面積保存性$\varphi^{*}(dy_2 \wedge dx_2) = d y_1 \wedge d x_1$と
        ツイスト性$\displaystyle \frac{\partial x_2}{\partial y_1} > 0$を持つものである.

        記号の濫用で, $\mathbb{S}^1 \times \mathbb{R}$の普遍被覆$\mathbb{R}^2$上の元もまた$(x, y)$のように書き, 
        また$\varphi$の$\mathbb{R}^2$への持ち上げも$\varphi$と書くことにすると, 
        面積保存ツイスト写像$\varphi$が特に$\mathbb{A}$上の完全シンプレクティック写像($\varphi^{*}(y dx) - y dx$が$\mathbb{A}$上の完全形式になる)とき, 
        $\varphi$の母関数と呼ばれる, 次の性質を満たす関数$h: \mathbb{R}^2 \rightarrow \mathbb{R}$が存在する:\\
        $h_1, h_2$をそれぞれ$h$の第$1$成分, 第$2$成分での偏微分としたときに, $h(x_1 + 1, x_2 + 1) = h(x_1, x_2), h_{12} < 0$であり, 
        さらに
        \[
            \left\{
            \begin{aligned}
                h_1(x_1, x_2) &= - y_1 \\
                h_2(x_1, x_2) &= y_2
            \end{aligned}
            \right.
            \iff \varphi(x_1, y_1) = (x_2, y_2)
        \]
        となる.\\
        (たぶん, $\mathbb{R}^2$で考える以上はポアンカレの補題から閉形式$\varphi^{*}(y dx) - y dx$が完全形式になることが保証されるけど, 
        アニュラス$\mathbb{A}$上でもなお$h$が意味を持つためには別で$\mathbb{A}$上で完全形式になることを保証しないといけない...のだと思う.)
        
        特に$\varphi$が母関数$h$を持つなら, $(x_1, y_1)$が与えられた下で, $\{(x_n, y_n)\}_{n}$が軌道$\{\varphi^{n}(x_1, y_1)\}_{n}$になることと
        \[
                h_2(x_{i-1}, x_{i}) + h_1(x_{i}, x_{i+1}) = 0, \ y_i = - h_2(x_{i}, x_{i+1}) \quad (\forall i \in \mathbb{Z})
        \]
        が同値であることがわかる.

    \section{不変曲線と差分方程式への簡約化}
        面積保存ツイスト写像$\varphi$の不変曲線$\gamma \subset \mathbb{A}$とは, 
        $\varphi$の不変集合であって$\mathbb{R}^2$上への持ち上げを$w(\theta) = (u(\theta), v(\theta))$としたときに
        $u(\theta) - \theta$および$v(\theta)$が周期$1$の周期関数となるものである.
        これは, $u(\theta + 1) - (\theta + 1) = u(\theta) - \theta$より$u(\theta + 1) - u(\theta) = 1$より, 
        アニュラス$\mathbb{A}$を$x$方向に一周して戻ってくる曲線であることを意味している.
        
        さて, ある回転数$\omega$についての不変曲線$\gamma$, つまり
        \[
                \varphi(w(\theta)) = w(\theta + \omega)
        \]
        を見つけたい. これはラグランジュ方程式と呼ばれることもある次の$2$階差分方程式
        \[
            E[u(\theta)] = h_1(u(\theta), u(\theta + \omega)) + h_2(u(\theta), u(\theta - \omega)) \equiv 0
        \]
        が解ければ, $v(\theta) = - h_1(u(\theta), u(\theta + \omega))$とおくことで不変曲線を見つけることができる.
        以下, $u^{+}(\theta) = u(\theta + \omega), u^{-}(\theta) = u(\theta - \omega)$とする.

        \begin{remark}
            あとで使うので, $u_{\theta} E[u]$の平均値$\displaystyle \int^{1}_{0} (u_{\theta} E[u])(\theta) d \theta = 0$を計算しておく.
            \[
                \frac{\partial h}{\partial \theta}(u, u^{+}) = u_{\theta} (h_1(u, u^{+}) + h_2(u, u^{+}))
            \]
            より, $\nabla f := f(\theta + \omega) - f(\theta)$と表記すれば,
            \begin{align}
                \frac{\partial h}{\partial \theta}(u, u^{+}) - u_{\theta} h_2(u^{-}, u)
                &= u_{\theta} (h_1(u, u^{+}) + h_2(u, u^{+}) - h_2(u, u^{+}) + h_2(u^{-}, u)) \\
                &= u_{\theta} (h_1(u, u^{+}) + h_2(u^{-}, u)) \\
                &= u_{\theta} E[u]
            \end{align}
            したがって, 
            \[
                u_{\theta} E[u] = \frac{\partial h}{\partial \theta}(u, u^{+}) - u_{\theta} h_2(u^{-}, u)
            \]
            と表すことができる.
            ここで, $h(x_1+1, x_2 + 1) = h(x_1, x_2)$であることと, 
            $f(\theta)$が周期$1$の周期関数であれば$\displaystyle \int^{1}_{0} (\nabla f)(\theta) d \theta = \int^{1}_{0} (f(\theta + \omega) - f(\theta)) d \theta = 0$であることから,
            結局
            \[
                \int^{1}_{0} (u_{\theta} E[u])(\theta) d \theta = 0
            \]
            である.

        \end{remark}

        \red{TO DO: ツイスト定理の証明をまとめる}
    \begin{thebibliography}{10}
    \nocite{*}
	\bibitem{LM01}  M. Levi and J. Moser, A Lagrangian proof of the invariant curve theorem for twist mappings, 
    Smooth ergodic theory and its applications (Seattle, WA, 1999), Proc. Sympos. Pure Math. \textbf{69}, 733-746, Amer. Math. Soc., Providence, RI, 2001 
\end{thebibliography}
\end{document}